\documentclass{article}

% Additional packages & macros
\usepackage[a4paper, margin=1in]{geometry}
\usepackage{fancyhdr}
\pagestyle{fancy}
\usepackage{minted}
\usepackage{xcolor}
\usemintedstyle{pastie}
\usepackage{booktabs}     % For cleaner tables (\toprule, \midrule, \bottomrule)
\usepackage{amsmath}      % General maths support
\usepackage{amssymb}      % Additional maths symbols
\usepackage{graphicx}     % If you're including any images
\usepackage{hyperref}     % For clickable links (if needed)

% Header and footer
% ! Unit name
\newcommand{\unitName}{CAB302 Software Development}
% ! Unit semester 
\newcommand{\unitTime}{Semester 2, 2025}
% ! Unit coordinator name
\newcommand{\unitCoordinator}{Unit Coordinator: Alessandro Soro}
% ! Document authors
\newcommand{\documentAuthors}{Monica Borg}

\fancyhead[L]{\unitName}
\fancyhead[R]{\leftmark}
\fancyfoot[C]{\thepage}

\date{}

\begin{document}
%
\begin{titlepage}
    \vspace*{\fill}
    \begin{center}
        \LARGE{\textbf{\unitName}} \\[0.1in]
        \normalsize{\unitTime} \\[0.2in]
        \normalsize\textit{\unitCoordinator} \\[0.2in]
        \documentAuthors
    \end{center}
    \vspace*{\fill}
    \thispagestyle{empty}
\end{titlepage}
\newpage
%
\tableofcontents
\newpage
%
\section{Object Oriented Programming with Java}

\subsection{Java}
Java is a high-level, class-based, object-oriented programming language that is platform-independent and widely used for developing applications across desktop, web, and mobile environments. It follows the syntax and structure of the C-family of languages, making it familiar to those with experience in C or C\#.

Originally created in the 1990s at Sun Microsystems, Java was designed to be platform-independent and architecture-neutral, enabling code to run consistently across various systems (regardless of endianness\footnote{Endianness, in computing, refers to the order in which bytes within a word of digital data are stored in computer memory or transmitted over a network. It can be either big-endian (most significant byte first) or little-endian (least significant byte first)}). It is fully object-oriented, supports automatic garbage collection for robustness, and was envisioned as a programming language for the Internet age. Java became open source in 2006 and has since been widely adopted as the primary platform for Android development.

A basic Java program includes a class definition and a \texttt{main} method, which serves as the entry point of execution. For example, a simple program that prints \texttt{Hello, world!} to the console would be written as:

\begin{minted}[fontsize=\small, linenos, breaklines]{java}
public class HelloWorld {
    public static void main(String[] args) {
        System.out.println("Hello, world!");
    }
}
\end{minted}

\subsection{Primitive Data Types in Java}

Java provides a set of built-in primitive data types for representing simple values. These types are not objects and serve as the most basic building blocks for data manipulation:

\begin{table}[h!]
\centering
\begin{tabular}{@{}lll@{}}
\toprule
\textbf{Type} & \textbf{Size} & \textbf{Description} \\
\midrule
\texttt{byte}   & 8-bit   & Signed two’s complement integer. Range: \texttt{-128} to \texttt{127} \\
\texttt{short}  & 16-bit  & Signed two’s complement integer. Range: \texttt{-32,768} to \texttt{32,767} \\
\texttt{int}    & 32-bit  & Signed two’s complement integer. Range: \texttt{-2,147,483,648} to \texttt{2,147,483,647} \\
\texttt{long}   & 64-bit  & Signed two’s complement integer. Range: \texttt{-9,223,372,036,854,775,808} \\
                &         & to \texttt{9,223,372,036,854,775,807} \\
\texttt{float}  & 32-bit  & IEEE 754 single-precision floating point \\
\texttt{double} & 64-bit  & IEEE 754 double-precision floating point \\
\texttt{boolean}& 1-bit*  & Represents \texttt{true} or \texttt{false} \\
\texttt{char}   & 16-bit  & A single Unicode character \\
\bottomrule
\end{tabular}
\end{table}
\textit{*Note: The size of a \texttt{boolean} is not fixed and depends on the JVM implementation. It is often 1 byte but not guaranteed.}

\subsection{Java Operators}

\subsubsection{Arithmetic Operators}
\begin{center}
\begin{tabular}{@{}lll@{}}
\toprule
\textbf{Operator} & \textbf{Name} & \textbf{Example} \\
\midrule
\texttt{+}  & Addition       & \texttt{x + y} \\
\texttt{-}  & Subtraction    & \texttt{x - y} \\
\texttt{*}  & Multiplication & \texttt{x * y} \\
\texttt{/}  & Division       & \texttt{x / y} \\
\texttt{\%} & Modulus        & \texttt{x \% y} \\
\texttt{++} & Increment      & \texttt{++x} \\
\texttt{--} & Decrement      & \texttt{--x} \\
\bottomrule
\end{tabular}
\end{center}

\vspace{1em}
\subsubsection{Assignment Operators}
\begin{center}
\begin{tabular}{@{}lll@{}}
\toprule
\textbf{Operator} & \textbf{Example} & \textbf{Equivalent To} \\
\midrule
\texttt{=}    & \texttt{x = 5}    & \texttt{x = 5} \\
\texttt{+=}   & \texttt{x += 3}   & \texttt{x = x + 3} \\
\texttt{-=}   & \texttt{x -= 3}   & \texttt{x = x - 3} \\
\texttt{*=}   & \texttt{x *= 3}   & \texttt{x = x * 3} \\
\texttt{/=}   & \texttt{x /= 3}   & \texttt{x = x / 3} \\
\texttt{\%=}  & \texttt{x \%= 3}  & \texttt{x = x \% 3} \\
\texttt{\&=}  & \texttt{x \&= 3}  & \texttt{x = x \& 3} \\
\texttt{|=}   & \texttt{x |= 3}   & \texttt{x = x | 3} \\
\texttt{\^=}  & \texttt{x \^= 3}  & \texttt{x = x \^ 3} \\
\texttt{>>=}  & \texttt{x >>= 3}  & \texttt{x = x >> 3} \\
\texttt{<<=}  & \texttt{x <<= 3}  & \texttt{x = x << 3} \\
\bottomrule
\end{tabular}
\end{center}

\vspace{1em}
\subsubsection{Logical Operators}
\begin{center}
\begin{tabular}{@{}lll@{}}
\toprule
\textbf{Operator} & \textbf{Name} & \textbf{Example} \\
\midrule
\texttt{\&\&} & Logical AND & \texttt{x < 5 \&\& x < 10} \\
\texttt{||}  & Logical OR  & \texttt{x < 5 || x < 4} \\
\texttt{!}   & Logical NOT & \texttt{!(x < 5 \&\& x < 10)} \\
\bottomrule
\end{tabular}
\end{center}

\vspace{1em}
\subsubsection{Comparison Operators}
\begin{center}
\begin{tabular}{@{}lll@{}}
\toprule
\textbf{Operator} & \textbf{Name} & \textbf{Example} \\
\midrule
\texttt{==} & Equal to                 & \texttt{x == y} \\
\texttt{!=} & Not equal                & \texttt{x != y} \\
\texttt{>}  & Greater than             & \texttt{x > y} \\
\texttt{<}  & Less than                & \texttt{x < y} \\
\texttt{>=} & Greater than or equal to & \texttt{x >= y} \\
\texttt{<=} & Less than or equal to    & \texttt{x <= y} \\
\bottomrule
\end{tabular}
\end{center}

\subsubsection{Operator Overloading}

Unlike some other languages like C++ or Python, Java does \textbf{not support operator overloading}. This means you cannot redefine the behaviour of standard operators (e.g., \texttt{+}, \texttt{-}) for user-defined types such as classes.

\subsection{Object-Oriented Principles in Java}

Java is a fully object-oriented language, and its design is built around four key principles:

\begin{itemize}
    \item \textbf{Encapsulation} – Combines data (fields) and behaviour (methods) into a single unit called a class. Access to data is controlled using access modifiers (e.g., \texttt{private}, \texttt{public}) to protect internal state.
    
    \item \textbf{Abstraction} – Hides internal implementation details and exposes only essential functionality through interfaces or abstract classes. This simplifies code usage and improves maintainability.
    
    \item \textbf{Inheritance} – Allows a new class (subclass) to inherit properties and behaviours from an existing class (superclass), enabling code reuse and logical hierarchy using the \texttt{extends} keyword.
    
    \item \textbf{Polymorphism} – Enables the same method call to behave differently depending on the object type at runtime. Achieved through method overriding and interface implementation.
\end{itemize}

\subsubsection{Everything's an Object}

In Java, \texttt{Object} is the root of the class hierarchy. Every class—either directly or indirectly—inherits from \texttt{java.lang.Object}, meaning all Java objects (including arrays) have access to its methods.

\begin{itemize}
    \item \texttt{Object clone()} – Creates and returns a shallow copy of the object (must implement \texttt{Cloneable}).
    \item \texttt{boolean equals(Object obj)} – Checks if another object is "equal to" this one.
    \item \texttt{int hashCode()} – Returns a hash code used in hash-based collections like \texttt{HashMap}.
    \item \texttt{String toString()} – Returns a string representation of the object (default: class name + hash).
    \item \texttt{Class<?> getClass()} – Returns the runtime class of the object.
    \item \texttt{void finalize()} – Called by the garbage collector before object destruction (deprecated in recent versions).
    \item \texttt{void wait()}, \texttt{wait(long)}, \texttt{wait(long, int)} – Causes a thread to wait until notified.
    \item \texttt{void notify()}, \texttt{void notifyAll()} – Wakes one or all threads waiting on the object’s monitor.
\end{itemize}

These methods form the foundation of Java’s object model, thread synchronisation, and identity/behaviour of all class instances.

\smallskip
\textit{See:} \href{https://docs.oracle.com/javase/8/docs/api/java/lang/Object.html}{\texttt{docs.oracle.com/javase/8/docs/api/java/lang/Object.html}}

\subsection{Using Interfaces in Java for OOP}

In Java, an \textbf{interface} defines a contract of method signatures that a class can implement. Interfaces support abstraction by separating what a class does from how it does it, promoting loose coupling and flexibility in object-oriented design.

Interfaces can contain:
\begin{itemize}
    \item Abstract method declarations (implicitly \texttt{public abstract})
    \item \texttt{default} methods (with implementation)
    \item \texttt{static} methods
    \item Constants (\texttt{public static final})
\end{itemize}

A class implements an interface using the \texttt{implements} keyword and must provide concrete implementations for all abstract methods.

\subsubsection*{Example: Interface and Implementation}

\begin{minted}[fontsize=\small, linenos, breaklines]{java}
interface Animal {
    void makeSound(); // abstract method
}

class Dog implements Animal {
    public void makeSound() {
        System.out.println("Woof!");
    }
}

class Cat implements Animal {
    public void makeSound() {
        System.out.println("Meow!");
    }
}

public class Main {
    public static void main(String[] args) {
        Animal a1 = new Dog();
        Animal a2 = new Cat();
        a1.makeSound(); // Woof!
        a2.makeSound(); // Meow!
    }
}
\end{minted}

This example shows \textbf{polymorphism} through interface references. Both \texttt{Dog} and \texttt{Cat} classes implement \texttt{Animal}, allowing us to treat them uniformly while invoking behaviour defined in their respective implementations.

\medskip
\textit{Interfaces are ideal when you want multiple unrelated classes to share a common behaviour without enforcing a class hierarchy.}



\subsection{Git Fundamentals}

\subsection{JavaFx \& the Graphical User Interface (GUI)}

\subsection{The Integrated Build \& Javadoc}

\subsection{Refactoring \& Design Patterns}

\subsection{Threading and APIs}

% ===================================================================================
\newpage
\section{Software Development}

\subsection{Collaborative Programming \& High Performing Teams}

\subsection{Agile \& Sprint Planning}

\subsection{Persistence}

\subsection{Test Driven Development}

\subsection{User Testing}


\end{document}
