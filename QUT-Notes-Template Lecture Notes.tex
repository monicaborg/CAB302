\documentclass{article}

% Additional packages & macros
\usepackage[a4paper, margin=1in]{geometry}
\usepackage{fancyhdr}
\pagestyle{fancy} % Enable fancy headers and footers
\usepackage{listings}
\usepackage{xcolor}

% Define colours
\definecolor{purple}{HTML}{C678DD}  % For method names like 'main'
\definecolor{green}{HTML}{98C379}   % For class names like 'HelloWorld'
\definecolor{blue}{HTML}{61AFEF}    % For types like String, System
\definecolor{lighttext}{HTML}{ABB2BF}
\definecolor{gray}{HTML}{5C6370}

% Java syntax highlighting
\lstdefinelanguage{Java}{
  morekeywords={abstract, assert, boolean, break, byte, case, catch, char, class, const,
    continue, default, do, double, else, enum, extends, final, finally, float, for,
    goto, if, implements, import, instanceof, int, interface, long, native, new,
    package, private, protected, public, return, short, static, strictfp, super,
    switch, synchronized, this, throw, throws, transient, try, void, volatile, while},
  sensitive=true,
  morecomment=[l]//,
  morecomment=[s]{/*}{*/},
  morestring=[b]",
}

% Styling
\lstset{
  language=Java,
  basicstyle=\ttfamily\small\color{black},             % Default text (args, .out, etc.)
  keywordstyle=\color{blue}\bfseries,                  % Keywords (public, static, class)
  commentstyle=\color{gray}\itshape,                   % Comments (grey)
  stringstyle=\color{green},                           % Strings ("Hello, World!")
  emph={HelloWorld}, emphstyle=\color{green},          % Class name
  emph={[2]main}, emphstyle={[2]\color{purple}},        % Method name
  emph={[3]String, System}, emphstyle={[3]\color{blue}},% Java standard classes
  numbers=left,                                         % ← Line numbers are back
  numberstyle=\tiny\color{gray},                       % Line number style (grey + small)
  numbersep=5pt,
  showstringspaces=false,
  tabsize=2,
  breaklines=true,
  frame=none,
  captionpos=b
}


% Header and footer
% ! Unit name
\newcommand{\unitName}{CAB302 Software Development}
% ! Unit semester 
\newcommand{\unitTime}{Semester 2, 2025}
% ! Unit coordinator name
\newcommand{\unitCoordinator}{Unit Coordinator: Alessandro Soro}
% ! Document authors
\newcommand{\documentAuthors}{Monica Borg}

\fancyhead[L]{\unitName}
\fancyhead[R]{\leftmark}
\fancyfoot[C]{\thepage}

\date{}

\begin{document}
%
\begin{titlepage}
    \vspace*{\fill}
    \begin{center}
        \LARGE{\textbf{\unitName}} \\[0.1in]
        \normalsize{\unitTime} \\[0.2in]
        \normalsize\textit{\unitCoordinator} \\[0.2in]
        \documentAuthors
    \end{center}
    \vspace*{\fill}
    \thispagestyle{empty}
\end{titlepage}
\newpage
%
\tableofcontents
\newpage
%
\section{Object Oriented Programming in Java}

\subsection{Java}
Java is a high-level, class-based, object-oriented programming language that is platform-independent and widely used for developing applications across desktop, web, and mobile environments. It follows the syntax and structure of the C-family of languages, making it familiar to those with experience in C or C\#.

A basic Java program includes a class definition and a \texttt{main} method, which serves as the entry point of execution. For example, a simple program that prints ``Hello, world!'' to the console would be written as:

\begin{lstlisting}
public class HelloWorld {
    public static void main(String[] args) {
        System.out.println("Hello, world!");
    }
}
\end{lstlisting}

\subsubsection{}
\subsection{Git}

\subsection{Graphical User Interface (GUI)}

\subsection{The Integrated Build & Javadoc}

\subsection{Refactoring & Design Patterns}

\subsection{Threading and APIs}

% ===================================================================================
\newpage
\section{Software Development}

\subsection{Collaborative Programming & High Performing Teams}

\subsection{Agile & Sprint Planning}

\subsection{Persistence}

\subsection{Test Driven Development}

\subsection{User Testing}


\end{document}
